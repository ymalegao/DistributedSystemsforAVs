\documentclass[fleqn,10pt]{olplainarticle}
% Use option lineno for line numbers 


\begin{document}

\subsection*{Related Works}

Research on different applications and protocols of distributed consensus in an Autonomous Vehicle space. 


\subsection{Decentralized Distributed Consensus Approaches}
Consensus among vehicles can be done by electing leaders, \cite{lee2025distributedconsensusalgorithmprioritizing}, which uses a Raft-inspired protocol. A view is formed upon arrival at the intersection, using a vision-based approach. 
Candidates vote on who the leader is, and if consensus is not reached or unreachable, deterministic vision-based ordering is used. This protocol does not handle byzantine nodes during consensus, but provides safety and liveness. External factors such as velocity, speed, and braking distance are not considered. 


A leadless approach, \cite{inproceedings}, is also possible, where Vehicles continually broadcast a "claim" message to the view. Here, a view is formed by a set of lurk distances, where a candidate can start to receive messages long before arriving at the intersection. However, it does not transmit until its distance from the intersection is less than the lurk distance.  

An agent approaching an intersection generates a claim based on its earliest arrival time, predicted velocity, and exit time. When a claim is made, it is for an estimated period of time that the individual was in the intersection. 
A vehicle can use its estimated arrival time to cancel its claim and let another car claim that time. 

From both \cite{inproceedings} and \cite{lee2025distributedconsensusalgorithmprioritizing}, concurrency is allowed in the intersection. Suppose a car A has gotten votes for a quorum to enter the intersection, and another car B in the intersection is not conflicting with its path. In that case, Car B will be allowed to enter the intersection at the same time, despite not having a winning consensus. 






\bibliography{sample}

\end{document}