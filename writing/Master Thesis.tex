\documentclass[fleqn,10pt]{olplainarticle}
% Use option lineno for line numbers 


\begin{document}

\subsection*{Related Works}

Research on different applications and protocols of distributed consensus in an Autonomous Vehicle space. 

\subsection{Survey Paper}

\begin{enumerate}


\item
\subsubsection{Decentralized vs Centralized }
The typical approach for CAV intersection is distributed or centralized. Centralized can be split into a query-based Intersection manager or an Assignment-based. 

IN QB-IM, the IM will accept or reject requests from a CAV, while in AB-IM the IM will assign a reservation to the CAV. 

AB-IM can achieve higher throughputs compared to QB-IM, but processing time is greater.

Centralized is dependent on infrastructure and introduces a single point of failure 

\item

\subsubsection{Collision Detection}
To detect a possible conflict that two CAVs may have at the intersection, existing works have proposed two approaches: 
a spatio-temporal occupancy map, where an occupancy map is tracked to see which blocks of the grid are occupied by a CAV

Another approach is keeping track of the CAV's trajectory where the expected path of two CAVs is used to determine the location at which two CAVs may have a conflict. This can be done offline, as the expected paths of CAVs are known, e.g., for going straight or making a turn.

\item

\subsubsection{Scheduling }
The process of deciding which CAV should cross the intersection first and which CAV should cross second and so on. this can be done by policy like FCFS or Heuristics

One cool heuristic approach that reminded me of Hambazi: Vehicles can bid currency to beat out other vehicles to get reservations for the intersection. In many cases, a vehicle has to pay for the reservation of vehicles in front of it, too, to clear the queue. \cite{Vasirani_2012}



Time synchronization, which has received less attention, is a fundamental part of the intersection management. 


Rest of the paper talks about safety and I guess more AV stuff rather than intersection management or distributed consensus.
Realized very late into reading that its not stricly about distributed consensus... but it was knowledgeable


\end{enumerate}


\subsection{Centralized Approaches}
\begin{enumerate}
    \item
In a reservation-based system proposed by \cite{dresner}, an intersection is controlled by the IM (intersection manager), which assigns reservations of space and time to each autonomous vehicle intending to cross the intersection. The IM is provided with data like vehicle ID, vehicle size, arrival time, arrival speed, type of turn, and can make decisions from all vehicles sending it that information. This is FCFS. 

\cite{Vasirani_2012} proposes an auction-based allocation policy for assigning reservations. In this system, the same parameters are given to the IM ( arrival speed, lane, and type of turn). The additional parameter passed in is the value of its bid, i.e., the amount of money that it is willing to pay for the requested reservation. Once the new bids are collected, they constitute the bid set. Then, the IM executes the algorithm for the winner determination problem (WDP), and the winner set is built. During the WDP algorithm execution, the auctioneer still accepts incoming bids, but they will only be included in the bid set of the next round.

There is a significant delay introduced with this combinatorial auction, especially under heavy traffic. Even if a single vehicle is approaching with no one in the intersection, it will face a delay due to the auction process. 





\end{enumerate}


\subsection{Decentralized Distributed Consensus Approaches}

\begin{enumerate}
    \item 

Consensus among vehicles can be done by electing leaders, \cite{lee2025distributedconsensusalgorithmprioritizing}, which uses a Raft-inspired protocol. A view is formed upon arrival at the intersection, using a vision-based approach. 
Candidates vote on who the leader is, and if consensus is not reached or unreachable, deterministic vision-based ordering is used. This protocol does not handle byzantine nodes during consensus, but provides safety and liveness. External factors such as velocity, speed, and braking distance are not considered. 


\item 
A leadless approach, \cite{inproceedings}, is also possible, where Vehicles continually broadcast a "claim" message to the view. Here, a view is formed by a set of lurk distances, where a candidate can start to receive messages long before arriving at the intersection. However, it does not transmit until its distance from the intersection is less than the lurk distance.  

An agent approaching an intersection generates a claim based on its earliest arrival time, predicted velocity, and exit time. When a claim is made, it is for an estimated period of time that the individual was in the intersection. 
A vehicle can use its estimated arrival time to cancel its claim and let another car claim that time. 


\item 
In \cite{MIRHELI2019161}, they propose a distributed coordinated signal-free intersection control logic. 

CAVs in this system use DSRC to continuously broadcast their predicted trajectories within a detection range (~650 ft). Each vehicle incorporates neighbors’ predictions into its own optimization, then updates and rebroadcasts. Through iterative averaging, they reach consensus on conflict-free trajectories without central coordination

\paragraph{Consensus}
Not leader election but a {numerical consensus}: vehicles broadcast predicted trajectories starting from a set distance and iteratively adjust until stable. 

\begin{itemize}
    \item \textbf{Init:} $t=0$, set of vehicles $N^0$, states $S_{ij}^0$.
    \item \textbf{Iterate:} collect neighbors’ predictions, solve local optimization, update by averaging (Powell), repeat until stable.
    \item \textbf{Implement:} apply first step of trajectory, rebroadcast rest.
\end{itemize}

\textbf{Safety:} guaranteed via collision-avoidance constraints (no near-crashes).  

\textbf{Liveness:} not formally proven; empirically converges in 5--10 iterations.



\item 
\textbf{Similarities}

From both \cite{inproceedings} and \cite{lee2025distributedconsensusalgorithmprioritizing}, concurrency is allowed in the intersection. Suppose a car A has gotten votes for a quorum to enter the intersection, and another car B in the intersection is not conflicting with its path. In that case, Car B will be allowed to enter the intersection at the same time, despite not having a winning consensus. 

In \cite{MIRHELI2019161}, concurrent occupancy of the intersection is allowed by multiple CAVs, as long as trajectory constraints guarantee safety.


\end{enumerate}

\subsection{Consensus with churn}




\bibliography{sample}

\end{document}